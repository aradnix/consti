%~\showframe \showgrid
\mainlanguage[es]
\setuppagenumbering[alternative=doublesided, location={bottom, right}]
\setupwhitespace[small]
\setupindenting[yes, next, small]
\definestructureconversionset[sectionnumbers][0,R,R,R,n,n,n][n]
\definestructureresetset[default][1,0,1,1,0][1]
\setuplabeltext[chapter=Título~, section=Sección~, subsection=Párrafo~, subsubsection=Artículo~]
\setupheads[indentnext=yes, sectionconversionset=sectionnumbers, align=nothyphenated]
\setuplists[label=yes]
\setuphead[part][style=\bfd, number=no, placehead=yes, align=center, before={\mbox{}\vfill},after={\vfill\mbox{}}, header=empty, footer=empty]
\setuphead[chapter][style=\bfa, page={yes, blank},
sectionsegments=chapter, number=yes, after={\blank}]
\setuplist[chapter][width=fit, stopper={. }]
\setuphead[section][style=\bf, sectionsegments=section]
\setuplist[section][width=fit, stopper={. }]
\setuphead[subsection][style=\bf, sectionsegments=subsection,]
\setuplist[subsection][width=fit, stopper={. }]
\setuphead[subsubsection][style=\bf, number=yes, alternative=text, sectionsegments=subsubsection, before={\blank}]
\setuplist[subsubsection][alternative=d, pagenumber=no]
\setupinteraction[state=start, color=, style=, focus=standard]
\enabledirectives[references.border]
\setuplayout[setups=*lessstrict]
\startsetups[*lessstrict]
    \setup[reset]
    \widowpenalty=10000
    \clubpenalty=10000
    \brokenpenalty=0
\stopsetups
\setupitemize[each][broad][margin=3em, itemalign=flushright]
\setupxtable[split=yes, frame=off]
\starttext

\startstandardmakeup
\startalignment[center]
  \blank[2*big]
    {\tfd\setupinterlinespace Constitución Política de la\\ República Mexicana\par}
  \blank[3*medium]
    {\tfa Congreso Extraordinario Constituyente}
  \blank[2*medium]
    {\tfa 12 de febrero de 1857}
  \blank[3*medium]
\stopalignment
\stopstandardmakeup

\part{Constitución Política de la República Mexicana de 1857}

\page[yes,blank,right]

\subject{Febrero 12 de 1857.- Constitucion Política de la República.}

Ignacio Comonfort, Presidente sustituto de la República Mexicana, á los habitantes de ella, sabed: Que el congreso extraordinario constituyente ha decretado lo que sigue:

En el nombre de Dios y con la autoridad del pueblo mexicano.

Los representantes de los diferentes Estados, del Distrito y Territorios que componen la República de México, llamados por el plan proclamado en Ayutla el 1 de Marzo de 1854, reformado en Acapulco el día 11 del mismo mes y año, y por la convocatoria expedida el 17 de Octubre de 1855, para constituir á la Nación bajo la forma de República democrática, representativa, popular, poniendo en ejercicio los poderes con que están investidos, cumplen con su alto encargo decretando la siguiente:

\subject{Constitución}

\startalign[center]
Política de la República Mexicana sobre la indestructible base de su legítima independencia, proclamada el 16 de setiembre de 1810, y consumada el 27 de setiembre de 1821.
\stopalign

\chapter{\mbox{}}

\section{De los derechos del hombre}

\subsubsection{\mbox{}} El pueblo mexicano reconoce, que los derechos del hombre son la base y el objeto de las instituciones sociales. En consecuencia, declara: que todas las leyes y todas las autoridades del país, deben respetar y sostener las garantías que otorga la presente Constitución.

\subsubsection{\mbox{}} En la República todos nacen libres. Los esclavos que pisen el territorio nacional, recobran, por ese solo hecho, su libertad, y tienen derecho á la proteccion de las leyes.

\subsubsection{\mbox{}} La enseñanza es libre. La ley determinará qué profesiones necesitan título para su ejercicio, y con qué requisitos se deben expedir.

\subsubsection{\mbox{}} Todo hombre es libre para abrazar la profesion, industria ó trabajo que le acomode, siendo útil y honesto, y para aprovecharse de sus productos. Ni uno ni otro se le podrá impedir sino por sentencia judicial cuando ataque los derechos de tercero, ó por resolucion gubernativa, dictada en los términos que marque la ley, cuando ofenda los de la sociedad.

\subsubsection{\mbox{}} Nadie puede ser obligado á prestar trabajos personales, sin la justa retribucion y sin su pleno consentimiento. La ley no puede autorizar ningun contrato que tenga por objeto la pérdida ó el irrevocable sacrificio de la libertad del hombre, ya sea por causa de trabajo, de educacion ó de voto religioso. Tampoco puede autorizar convenios en que el hombre pacte su proscripcion ó destierro.

\subsubsection{\mbox{}} La manifestacion de las ideas no puede ser objeto de ninguna inquisicion judicial ó administrativa, sino en el caso de que ataque la moral, los derechos de tercero, provoque á algun crímen ó delito, ó perturbe el órden público.

\subsubsection{\mbox{}} Es inviolable la libertad de escribir y publicar escritos sobre cualquiera materia. Ninguna ley ni autoridad puede establecer la prévia censura, ni exigir fianza á los autores ó impresores, ni coartar la libertad de imprenta, que no tiene más límites que el respeto á la vida privada, á la moral y á la paz pública. Los delitos de imprenta serán juzgados por un jurado que califique el hecho, y por otro que aplique la ley y designe la pena.

\subsubsection{\mbox{}} Es inviolable el derecho de peticion ejercido por escrito, de una manera pacífica y respetuosa; pero en materias políticas sólo pueden ejercerlo los ciudadanos de la República. á toda peticion debe recaer un acuerdo escrito de la autoridad á quien se haya dirigido, y ésta tiene obligacion de hacer conocer el resultado al peticionario.

\subsubsection{\mbox{}} á nadie se le puede coartar el derecho de asociarse ó de reunirse pacíficamente con cualquier objeto lícito; pero solamente los ciudadanos de la República pueden hacerlo para tomar parte en los asuntos políticos del país. Ninguna reunion armada tiene derecho de deliberar.

\subsubsection{\mbox{}} Todo hombre tiene derecho de poseer y portar armas para su seguridad y legítima defensa. La ley señalará cuáles son las prohibidas y la pena en que incurren los que las portaren.

\subsubsection{\mbox{}} Todo hombre tiene derecho para entrar y salir de la República, viajar por su territorio y mudar de residencia sin necesidad de carta de seguridad, pasaporte, salvo-conducto ú otro requisito semejante. El ejercicio de este derecho no perjudica las legítimas facultades de la autoridad judicial ó administrativa, en los casos de responsabilidad criminal ó civil.

\subsubsection{\mbox{}} No hay, ni se reconocen en la República, títulos de nobleza, ni prerrogativas, ni honores hereditarios. Sólo el pueblo, legítimamente representado, puede decretar recompensas en honor de los que hayan prestado ó prestaren servicios eminentes á la patria ó á la humanidad.

\subsubsection{\mbox{}} En la República Mexicana nadie puede ser juzgado por leyes privativas, ni por tribunales especiales. Ninguna persona ni corporacion, puede tener fueros, ni gozar emolumentos que no sean compensacion de un servicio público, y estén fijados por la ley. Subsiste el fuero de guerra solamente para los delitos y faltas que tengan exacta conexion con la disciplina militar. La ley fijará con toda claridad los casos de excepcion.

\subsubsection{\mbox{}} No se podrá expedir ninguna ley retroactiva. Nadie puede ser juzgado ni sentenciado, sino por leyes dadas con anterioridad al hecho y exactamente aplicadas á él, por el tribunal que préviamente haya establecido la ley.

\subsubsection{\mbox{}} Nunca se celebrarán tratados para la extradicion de reos políticos, ni para la de aquellos delincuentes del órden comun que hayan tenido en el país en donde cometieron el delito la condicion de esclavos; ni convenios ó tratados en virtud de los que se alteren las garantías y derechos que esta Constitucion otorga al hombre y al ciudadano.

\subsubsection{\mbox{}} Nadie puede ser molestado en su persona, familia, domicilio, papeles y posesiones, sino en virtud de mandamiento escrito de la autoridad competente, que funde y motive la causa legal del procedimiento. En el caso de delito infraganti, toda persona puede aprehender al delincuente y á sus cómplices, poniéndolos sin demora á disposicion de la autoridad inmediata.

\subsubsection{\mbox{}} Nadie puede ser preso por deudas de un carácter puramente civil. Nadie puede ejercer violencia para reclamar su derecho. Los tribunales estarán siempre expeditos para administrar justicia. Esta será gratuita, quedando en consecuencia abolidas las costas judiciales.

\subsubsection{\mbox{}} Solo habrá lugar á prision por delito que merezca pena corporal. En cualquier estado del proceso en que aparezca que al acusado no se le puede imponer tal pena, se pondrá en libertad bajo de fianza. En ningun caso podrá prolongarse la prision ó detencion por falta de pago de honorarios, ó de cualquiera otra ministracion de dinero.

\subsubsection{\mbox{}} Ninguna detencion podrá exceder del término de tres dias, sin que se justifique con un auto motivado de prision y los demás requisitos que establezca la ley. El solo lapso de este término, constituye responsables á la autoridad que la ordena ó consiente, y á los agentes, ministros, alcaides ó carceleros que la ejecuten. Todo maltratamiento en la aprehension ó en las prisiones, toda molestia que se infiera sin motivo legal, toda gabela ó contribucion en las cárceles, es un abuso que deben corregir las leyes y castigar severamente las autoridades.

\subsubsection{\mbox{}} En todo juicio criminal, el acusado tendrá las siguientes garantías:
\startitemize[R]
\item Que se le haga saber el motivo del procedimiento y el nombre de cuarenta y ocho horas, contadas desde que esté á disposicion de su juez.
\item Que se le caree con los testigos que depongan en su contra.
\item Que se le faciliten los datos que necesite y consten en el proceso, para preparar sus descargos.
\item Que se le oiga en defensa por sí ó por persona de su confianza, ó por ambos, segun su voluntad. En caso de no tener quien lo defienda, se le presentará lista de los defensores de oficio, para que elija el que, ó los que le convengan.
\stopitemize

\subsubsection{\mbox{}} La aplicacion de las penas propiamente tales, es exclusiva de la autoridad judicial. La política ó administrativa solo podrá imponer como correccion, hasta quinientos pesos de multa, ó hasta un mes de reclusion, en los casos y modo que expresamente determine la ley.

\subsubsection{\mbox{}} Quedan para siempre prohibidas las penas de mutilacion y de infamia, la marca, los azotes, los palos, el tormento de cualquiera especie, la multa excesiva, la confiscacion de bienes y cualesquiera otras penas inusitadas ó trascendentales.

\subsubsection{\mbox{}} Para la abolicion de la pena de muerte queda á cargo del poder administrativo el establecer, á la mayor brevedad, el régimen penitenciario. Entre tanto, queda abolida para los delitos políticos, y no podrá extenderse á otros casos más que al traidor á la patria en guerra extranjera, al salteador de caminos, al incendiario, al parricida, al homicida con alevosía, premeditacion ó ventaja, á los delitos graves del órden militar y á los de piratería que definiere la ley.

\subsubsection{\mbox{}} Ningun juicio criminal puede tener más de tres instancias. Nadie puede ser juzgado dos veces por el mismo delito, ya sea que en el juicio se le absuelva ó se le condene. Queda abolida la práctica de absolver de la instancia.

\subsubsection{\mbox{}} La correspondencia que bajo cubierta circule por las estafetas, está libre de todo registro. La violacion de esta garantía es un atentado que la ley castigará severamente.

\subsubsection{\mbox{}} En tiempo de paz ningun militar puede exigir alojamiento, bagaje ni otro servicio real ó personal, sin el consentimiento del propietario. En tiempo de guerra sólo podrá hacerlo en los términos que establezca la ley.

\subsubsection{\mbox{}} La propiedad de las personas no puede ser ocupada sin su consentimiento, sino por causa de utilidad pública y prévia indemnizacion. La ley determinará la autoridad que deba hacer la expropiacion y los requisitos con que ésta haya de verificarse.

Ninguna corporacion civil ó eclesiástica, cualquiera que sea su carácter, denominacion ú objeto, tendrá capacidad legal para adquirir en propiedad ó administrar por sí bienes raíces, con la única excepcion de los edificios destinados inmediata y directamente al servicio ú objeto de la institucion.

\subsubsection{\mbox{}} No habrá monopolios ni estancos de ninguna clase, ni prohibiciones á título de proteccion á la industria. Exceptúanse únicamente, los relativos á la acuñacion de moneda, á los correos, á los privilegios que, por tiempo limitado, conceda la ley á los inventores ó perfeccionadores de alguna mejora.

\subsubsection{\mbox{}} En los casos de invasion, perturbacion grave de la paz pública, ó cualesquiera otros que pongan á la sociedad en grande peligro ó conflicto, solamente el presidente de la República, de acuerdo con el consejo de ministros y con aprobacion del congreso de la Union, y, en los recesos de éste, de la diputacion permanente, puede suspender las garantías otorgadas en esta Constitucion, con excepcion de las que aseguran la vida del hombre; pero deberá hacerlo por un tiempo limitado, por medio de prevenciones generales y sin que la suspension pueda contraerse á determinado individuo.

Si la suspension tuviere lugar hallándose el congreso reunido, éste concederá las autorizaciones que estime necesarias para que el ejecutivo haga frente á la situacion. Si la suspension se verificare en tiempo de receso, la diputacion permanente convocará sin demora al congreso para que las acuerde.

\section{De los mexicanos}

\subsubsection{\mbox{}} Son mexicanos:
\startitemize[R]
\item Todos los nacidos dentro ó fuera del territorio de la República, de padres mexicanos.
\item Los extranjeros que se naturalicen conforme á las leyes de la federación.
\item Los extranjeros que adquieran bienes raíces en la República ó tengan hijos mexicanos, siempre que no manifiesten
la resolucion de conservar su nacionalidad.
\stopitemize

\subsubsection{\mbox{}} Es obligacion de todo mexicano:
\startitemize[R]
\item Defender la independencia, el territorio, el honor, los derechos é intereses de su patria.
\item Contribuir para los gastos públicos, así de la federacion como del Estado y municipio en que resida, de la manera proporcional y equitativa que dispongan las leyes.
\stopitemize

\subsubsection{\mbox{}} Los mexicanos serán preferidos á los extranjeros, en igualdad de circunstancias, para todos los empleos, cargos ó comisiones de nombramiento de las autoridades, en que no sea indispensable la calidad de ciudadanos. Se expedirán leyes para mejorar la condicion de los mexicanos laboriosos, premiando á los que se distingan en cualquier ciencia ó arte, estimulando al trabajo y fundando colegios y escuelas prácticas de artes y oficios.

\section{De los extranjeros}

\subsubsection{\mbox{}} Son extranjeros los que no posean las calidades determinadas en el art. 30. Tienen derecho á las garantías otorgadas en la seccion 1a título 1° de la presente Constitucion, salva (sic) en todo caso la facultad que el gobierno tiene para expeler al extranjero pernicioso. Tienen obligacion de contribuir para los gastos públicos, de la manera que dispongan las leyes, y de obedecer y respetar las instituciones, leyes y autoridades del país, sujetándose á los fallos y sentencias de los tribunales, sin poder intentar otros recursos, que los que las leyes conceden á los mexicanos.

\section{De los ciudadanos mexicanos}

\subsubsection{\mbox{}} Son ciudadanos de la República todos los que, teniendo la calidad de mexicanos, reunan además las siguientes:
\startitemize[R]
\item Haber cumplido diez y ocho años siendo casados, ó veintiuno si no lo son.
\item Tener un modo honesto de vivir.
\stopitemize

\subsubsection{\mbox{}} Son prerogativas del ciudadano:
\startitemize[R]
\item Votar en las elecciones populares.
\item Poder ser votado para todos los cargos de eleccion popular, y nombrado para cualquier otro empleo ó comision, teniendo las calidades que la ley establezca.
\item Asociarse para tratar los asuntos políticos del país.
\item Tomar las armas en el ejército ó en la guardia nacional, para la defensa de la República y de sus instituciones.
\item Ejercer en toda clase de negocios el derecho de peticion.
\stopitemize

\subsubsection{\mbox{}} Son obligaciones del ciudadano de la República:
\startitemize[R]
\item Inscribirse en el padron de su municipalidad, manifestando la propiedad que tiene, ó la industria, profesion ó trabajo de que subsiste.
\item Alistarse en la guardia nacional.
\item Votar en las elecciones populares, en el distrito que le corresponda.
\item Desempeñar los cargos de eleccion popular de la federacion, que en ningun caso serán gratuitos.
\stopitemize

\subsubsection{\mbox{}} La calidad de ciudadano se pierde:
\startitemize[R]
\item Por naturalizacion en país extranjero.
\item Por servir oficialmente al gobierno de otro país, ó admitir de él condecoraciones, títulos ó funciones, sin prévia licencia del congreso federal. Exceptúanse los títulos literarios, científicos y humanitarios, que pueden aceptarse libremente.
\stopitemize

\subsubsection{\mbox{}} La ley fijará los casos y la forma en que se pierden ó suspenden los derechos de ciudadano, y la manera de hacer la rehabilitacion.

\chapter{\mbox{}}

\section{De la soberanía nacional y de la forma de gobierno}

\subsubsection{\mbox{}} La soberanía nacional reside esencial y originariamente en el pueblo. Todo poder público dimana del pueblo y se instituye para su beneficio. El pueblo tiene en todo tiempo el inalienable derecho de alterar ó modificar la forma de su gobierno.

\subsubsection{\mbox{}} Es voluntad del pueblo mexicano constituirse en una república representativa, democrática, federal, compuesta de Estados libres y soberanos en todo lo concerniente á su régimen interior, pero unidos en una federacion establecida segun los principios de esta ley fundamental.

\subsubsection{\mbox{}} El pueblo ejerce su soberanía por medio de los poderes de la Union en los casos de su competencia, y por los de los Estados para lo que toca á su régimen interior, en los términos respectivamente establecidos por esta Constitucion federal y las particulares de los Estados, las que en ningun caso podrán contravenir á las estipulaciones del pacto federal.

\section{De las partes integrantes de la federacion y del territorio nacional}

\subsubsection{\mbox{}} El territorio nacional comprende el de las partes integrantes de la federacion, y además el de las islas adyacentes en ambos mares.

\subsubsection{\mbox{}} Las partes integrantes de la federacion, son: los Estados de Aguascalientes, Colima, Chiapas, Chihuahua, Durango, Guanajuato, Guerrero, Jalisco, México, Michoacan, Nuevo-Leon y Coahuila, Oaxaca, Puebla, Querétaro, San Luis Potosí, Sinaloa, Sonora, Tabasco, Tamaulipas, Tlaxcala, Valle de México, Veracruz, Yucatan, Zacatecas y el Territorio de la Baja California.

\subsubsection{\mbox{}} Los Estados de Aguascalientes, Chiapas, Chihuahua, Durango, Guerrero, México, Puebla, Querétaro, Sinaloa, Sonora, Tamaulipas y el Territorio de la Baja California, conservarán los límites que actualmente tienen.

\subsubsection{\mbox{}} Los Estados de Colima y Tlaxcala conservarán, en su nuevo carácter de Estados, los límites que han tenido como territorios de la federacion.

\subsubsection{\mbox{}} El Estado del Valle de México se formará del territorio que en la actualidad comprende el Distrito federal; pero la ereccion solo tendrá efecto, cuando los supremos poderes federales se trasladen á otro lugar.

\subsubsection{\mbox{}} El Estado de Nuevo Leon y Coahuila comprenderá el territorio que ha pertenecido á los dos distintos Estados que hoy lo forman, separándose la parte de la hacienda de Bonanza, que se reincorporará á Zacatecas, en los mismos términos en que estaba ántes de su incorporacion á Coahuila.

\subsubsection{\mbox{}} Los Estados de Guanajuato, Jalisco, Michoacan, Oaxaca, San Luis Potosí, Tabasco, Veracruz, Yucatan y Zacatecas, recobrarán la extension y límites que tenian en 31 de Diciembre de 1852, con las alteraciones que establece el artículo siguiente.

\subsubsection{\mbox{}} El pueblo de Contepec, que ha pertenecido á Guanajuato, se incorporará á Michoacan. La municipalidad de Ahualulco, que ha pertenecido á Zacatecas, se incorporará á San Luis Potosí. Las municipalidades de Ojocaliente y San Francisco de los Adames, que han pertenecido á San Luis, así como los pueblos de Nueva Tlaxcala y San Andrés del Teul, que han pertenecido á Jalisco, se incorporarán á Zacatecas. El departamento de Tuxpan continuará formando parte de Veracruz. El canton de Huimanguillo, que ha pertenecido á Veracruz, se incorporará á Tabasco.

\chapter{De la division de poderes}

\subsubsection{\mbox{}} El Supremo poder de la federacion se divide para su ejercicio en legislativo, ejecutivo y judicial. Nunca podrán reunirse dos ó más de estos poderes en una persona ó corporacion, ni depositarse el legislativo en un individuo.

\section{Del poder legislativo}

\subsubsection{\mbox{}} Se deposita el ejercicio del supremo poder legislativo en una asamblea que se denominará Congreso de la Union.

\subsection{De la eleccion é instalacion del congreso}

\subsubsection{\mbox{}} El Congreso de la Union se compondrá de representantes, elegidos en su totalidad cada dos años por los ciudadanos mexicanos.

\subsubsection{\mbox{}} Se nombrará un diputado por cada cuarenta mil habitantes, ó por una fraccion que pase de veinte mil. El territorio en que la poblacion sea menor de la que se fija en este artículo, nombrará sin embargo un diputado.

\subsubsection{\mbox{}} Por cada diputado propietario se nombrará un suplente.

\subsubsection{\mbox{}} La eleccion para diputado será indirecta en primer grado, y en escrutinio secreto, en los términos que disponga la ley electoral.

\subsubsection{\mbox{}} Para ser diputado se requiere: ser ciudadano mexicano en ejercicio de sus derechos; tener veinticinco años cumplidos el dia de la apertura de las sesiones; ser vecino del Estado ó Territorio que hace la eleccion; y no pertenecer al estado eclesiástico. La vecindad no se pierde por ausencia en desempeño de cargo público de eleccion popular.

\subsubsection{\mbox{}} El cargo de diputado es incompatible con cualquiera comision ó destino de la Union en que se disfrute sueldo.

\subsubsection{\mbox{}} Los diputados propietarios desde el dia de su eleccion, hasta el dia en que concluyan su encargo, no pueden aceptar ningun empleo de nombramiento del Ejecutivo de la Union, por el que se disfrute sueldo, sin prévia licencia del congreso. El mismo requisito es necesario para los diputados suplentes, que estén en ejercicio de sus funciones.

\subsubsection{\mbox{}} Los diputados son inviolables por sus opiniones manifestadas en el desempeño de su encargo, y jamás podrán ser reconvenidos por ellas.

\subsubsection{\mbox{}} El congreso califica las elecciones de sus miembros y resuelve las dudas que ocurran sobre ellas.

\subsubsection{\mbox{}} El congreso no puede abrir sus sesiones, ni ejercer su encargo, sin la concurrencia de más de la mitad del número total de sus miembros; pero los presentes deberán reunirse el dia señalado por la ley y compeler á los ausentes, bajo las penas que ella designe.

\subsubsection{\mbox{}} El congreso tendrá cada año dos períodos de sesiones ordinarias: el primero comenzará el 16 de Setiembre y terminará el 15 de Diciembre; y el segundo, improrogable, comenzará el 1o de Abril y terminará el último de Mayo.

\subsubsection{\mbox{}} A la apertura de sesiones del congreso asistirá el presidente de la Union, y pronunciará un discurso en que manifieste el estado que guarda el país. El presidente del congreso contestará en términos generales.

\subsubsection{\mbox{}} Toda resolucion del congreso no tendrá otro carácter que el de la ley ó acuerdo económico. Las leyes se comunicarán al Ejecutivo firmadas por el presidente y dos secretarios, y los acuerdos económicos por sólo dos secretarios.

\subsection{De la iniciativa y formacion de las leyes}

\subsubsection{\mbox{}} El derecho de iniciar leyes compete:
\startitemize[R]
\item Al Presidente de la Union.
\item A los diputados al Congreso Federal.
\item A las Legislaturas de los Estados.
\stopitemize

\subsubsection{\mbox{}} Las iniciativas presentadas por el presidente de la República, las legislaturas de los Estados ó las diputaciones de los mismos, pasarán desde luego á comision. Las que presentaren los diputados, se sujetarán á los trámites que designe el reglamento de debates.

\subsubsection{\mbox{}} Todo proyecto de ley que fuere desechado por el congreso, no podrá volver á presentarse en las sesiones del año.

\subsubsection{\mbox{}} El segundo período de sesiones se destinará, de toda preferencia, al exámen y votacion de los presupuestos del año fiscal siguiente; á decretar las contribuciones para cubrirlos y á la revision de la cuenta del año anterior, que presente el Ejecutivo.

\subsubsection{\mbox{}} El dia penúltimo del primer período de sesiones, presentará el Ejecutivo al congreso el proyecto de presupuesto del año próximo venidero y la cuenta del año anterior. Uno y otra pasarán á una comision compuesta de cinco representantes nombrados en el mismo dia, la cual tendrá la obligacion de examinar ambos documentos y presentar el dictámen sobre ellos, en la segunda sesion del segundo período.

\subsubsection{\mbox{}} Las iniciativas ó proyectos de ley deberán sujetarse á los trámites siguientes:
\startitemize[R]
\item Dictámen de comision.
\item Una ó dos discusiones en los términos que expresan las fracciones siguientes.
\item La primera discusion se verificará en el dia que designe el presidente del congreso, conforme á reglamento.
\item Concluida esta discusion se pasará al ejecutivo copia del expediente, para que en el término de siete dias manifieste su opinion, ó exprese que no usa de esa facultad.
\item Si la opinion del Ejecutivo fuere conforme, se procederá sin más discusion, á la votacion de la ley.
\item Si dicha opinion discrepare en todo ó en parte, volverá el expediente á la comision, para que, con presencia de las observaciones del gobierno, examine de nuevo el negocio.
\item El nuevo dictámen sufrirá nueva discusion, y concluida ésta se procederá á la votacion.
\item Aprobacion de la mayoría absoluta de los diputados presentes.
\stopitemize

\subsubsection{\mbox{}} En el caso de urgencia notoria, calificada por el voto de dos tercios de los diputados presentes, el congreso puede estrechar ó dispensar los trámites establecidos en el artículo 70.

\subsection{De las facultades del congreso}

\subsubsection{\mbox{}} El congreso tiene facultad:
\startitemize[R]
\item Para admitir nuevos Estados ó Territorios á la Union federal, incorporándolos á la nacion.
\item Para erigir los Territorios en Estados cuando tengan una poblacion de ochenta mil habitantes, y los elementos necesarios para proveer á su existencia política.
\item Para formar nuevos Estados dentro de los límites de los existentes, siempre que lo pida una poblacion de ochenta mil habitantes, justificando tener los elementos necesarios para proveer á su existencia política. Oirá en todo caso á las legislaturas de cuyo Territorio se trate, y su acuerdo solo tendrá efecto, si lo ratifica la mayoría de las legislaturas de los Estados.
\item Para arreglar definitivamente los límites de los Estados, terminando las diferencias que entre ellos se susciten sobre demarcacion de sus respectivos territorios, ménos cuando esas diferencias tengan un carácter contencioso.
\item Para cambiar la residencia de los supremos poderes de la federacion.
\item Para el arreglo interior del Distrito federal y Territorios, teniendo por base el que los ciudadanos elijan popularmente las autoridades políticas, municipales y judiciales, designándoles rentas para cubrir sus atenciones locales.
\item Para aprobar el presupuesto de los gastos de la federacion que anualmente debe presentarle el Ejecutivo, é imponer las contribuciones necesarias para cubrirlo.
\item Para dar bases bajo las cuales el Ejecutivo pueda celebrar empréstitos sobre el crédito de la nacion; para aprobar esos mismos empréstitos, y para reconocer y mandar pagar la deuda nacional.
\item Para expedir aranceles sobre el comercio extranjero, y para impedir, por medio de bases generales, que en el comercio de Estado á Estado, se establezcan restricciones onerosas.
\item Para establecer las bases generales de la legislacion mercantil.
\item Para crear y suprimir empleos públicos de la federación: señalar, sumentar (sic) ó disminuir sus dotaciones.
\item Para ratificar los nombramientos que haga el Ejecutivo de los ministros, agentes diplomáticos y cónsules, de los empleados superiores de hacienda, de los coroneles y demás oficiales superiores del ejército y armada nacional.
\item Para aprobar los tratados, convenios ó convenciones diplomáticas que celebre el Ejecutivo.
\item Para declarar la guerra en vista de los datos que le presente el Ejecutivo.
\item Para reglamentar el modo en que deban expedirse las patentes de corso, para dictar leyes, segun las cuales deban declararse buenas ó malas las presas de mar y tierra, y para expedir las relativas al derecho marítimo de paz y guerra.
\item Para conceder ó negar la entrada de tropas extranjeras en el Territorio de la federación; y consentir la estacion de escuadras de otra potencia, por más de un mes, en las aguas de la República.
\item Para permitir la salida de tropas nacionales fuera de los límites de la República.
\item Para levantar y sostener el ejército y la armada de la Union, y para reglamentar su organizacion y servicio.
\item Para dar reglamentos con el objeto de organizar, armar y disciplinar la guardia nacional; reservando á los ciudadanos que la formen, el nombramiento respectivo de jefes y oficiales, y á los Estados la facultad de instruirla conforme á la disciplina prescrita por dichos reglamentos.
\item Para dar su consentimiento á fin de que el ejecutivo pueda disponer de la guardia nacional, fuera de sus respectivos Estados ó Territorios, fijando la fuerza necesaria.
\item Para dictar leyes sobre naturalizacion, colonizacion y ciudadanía.
\item Para dictar leyes sobre vías generales de comunicacion y sobre postas y correos.
\item Para establecer casas de moneda, fijar las condiciones que esta deba tener, determinar el valor de la extranjera y adoptar un sistema general de pesos y medidas.
\item Para fijar las reglas á que debe sujetarse la ocupacion y enajenacion de terrenos baldíos y el precio de éstos.
\item Para conceder amnistías por delitos cuyo conocimiento
pertenezca á los tribunales de la federación.
\item Para conceder premios y recompensas por servicios eminentes prestados á la patria ó á la humanidad, y privilegios por tiempo ilimitado á los inventores ó perfeccionadores de alguna mejora.
\item Para prorogar por treinta dias útiles el primer período de sus sesiones ordinarias.
\item Para formar su reglamento interior y tomar las providencias necesarias para hacer concurrir á los diputados ausentes, y corregir las faltas ú omisiones de los presentes.
\item Para nombrar y remover libremente á los empleados de su secretaría y á los de la contaduría mayor, que se organizará segun lo disponga la ley.
\item Para expedir todas las leyes que sean necesarias y propias para hacer efectivas las facultades antecedentes y todas las otras concedidas por esta Constitucion á los poderes de la Union.
\stopitemize

\subsection{De la diputacion permanente}

\subsubsection{\mbox{}} Durante los recesos del Congreso de la Union, habrá una diputacion permanente, compuesta de un diputado por cada Estado y Territorio, que nombrará el congreso la víspera de la clausura de sus sesiones.

\subsubsection{\mbox{}} Las atribuciones de la diputacion permanente son las siguientes:
\startitemize[R]
\item Prestar su consentimiento para el uso de la guardia nacional, en los casos de que habla el art. 72, fraccion 20.
\item Acordar por sí sola, ó á peticion del Ejecutivo, la convocacion del congreso á sesiones extraordinarias.
\item Aprobar en su caso los nombramientos á que se refiere el art. 85, fraccion 3a.
\item Recibir el juramento al presidente de la República y á los ministros de la Suprema Corte de Justicia, en los casos prevenidos por esta Constitucion.
\item Dictaminar sobre todos los asuntos que queden sin resolucion en los expedientes, á fin de que la legislatura que sigue tenga desde luego de que ocuparse.
\stopitemize

\section{Del poder Ejecutivo}

\subsubsection{\mbox{}} Se deposita el ejercicio del supremo poder Ejecutivo de la Union, en un solo individuo que se denominará «Presidente de los Estados-Unidos Mexicanos».

\subsubsection{\mbox{}} La eleccion de presidente será indirecta en primer grado y en escrutinio secreto, en los términos que disponga la ley electoral.

\subsubsection{\mbox{}} Para ser presidente se requiere: ser ciudadano mexicano por nacimiento, en ejercicio de sus derechos, de treinta y cinco años cumplidos al tiempo de la eleccion, no pertenecer al estado eclesiástico y residir en el país al tiempo de verificarse la eleccion.

\subsubsection{\mbox{}} El presidente entrará á ejercer sus funciones el primero de Diciembre, y durará en su encargo cuatro años.

\subsubsection{\mbox{}} En las faltas temporales del presidente de la República, y en la absoluta, mientras se presenta el nuevamente electo, entrará á ejercer el poder, el presidente de la Suprema Corte de Justicia.

\subsubsection{\mbox{}} Si la falta del presidente fuere absoluta, se procederá á nueva eleccion con arreglo á lo dispuesto en el art. 76, y el nuevamente electo, ejercerá sus funciones hasta el dia último de Noviembre del cuarto año siguiente al de su eleccion.

\subsubsection{\mbox{}} El cargo de presidente de la Union, solo es renunciable por causa grave, calificada por el congreso, ante quien se presentará la renuncia.

\subsubsection{\mbox{}} Si por cualquier motivo la eleccion de presidente no estuviere hecha y publicada para el 1o de Diciembre en que debe verificarse el reemplazo, ó el electo no estuviere pronto á entrar en el ejercicio de sus funciones, cesará sin embargo el antiguo, y el supremo poder Ejecutivo se depositará interinamente en el presidente de la Suprema Corte de Justicia.

\subsubsection{\mbox{}} El presidente al tomar posesion de su encargo, jurará ante el congreso, y en su receso ante la diputacion permanente, bajo la fórmula siguiente: «Juro desempeñar leal y patrióticamente el encargo de presidente de los Estados-Unidos Mexicanos, conforme á la Constitucion, y mirando en todo por el bien y prosperidad de la Union».

\subsubsection{\mbox{}} El presidente no puede separarse del lugar de la residencia de los poderes federales, ni del ejercicio de sus funciones, sin motivo grave calificado por el congreso, y en sus recesos por la diputacion permanente.

\subsubsection{\mbox{}} Las facultades y obligaciones del presidente, son las siguientes:
\startitemize[R]
\item Promulgar y ejecutar las leyes que expida el congreso de la Union, proveyendo en la esfera administrativa á su exacta observancia.
\item Nombrar y remover libremente á los secretarios del despacho, remover á los agentes diplomáticos y empleados superiores de hacienda, y nombrar y remover libremente á los demás empleados de la Union, cuyo nombramiento ó remocion no estén determinados de otro modo en la Constitucion ó en las leyes.
\item Nombrar los ministros, agentes diplomáticos y cónsules generales, con aprobacion del congreso, y en sus recesos de la diputacion permanente.
\item Nombrar con aprobacion del congreso, los coroneles y demás oficiales superiores del ejército y armada nacional y los empleados superiores de hacienda.
\item Nombrar los demás oficiales del ejército y armada nacional, con arreglo á las leyes.
\item Disponer de la fuerza armada permanente de mar y tierra para la seguridad interior y defensa exterior de la federación.
\item Disponer de la guardia nacional para los mismos objetos, en los términos que previene la fraccion 20 del art. 72.
\item Declarar la guerra en nombre de los Estados-Unidos Mexicanos, prévia ley del congreso de la Union.
\item Conceder patentes de corso con sujecion á las bases fijadas por el congreso.
\item Dirigir las negociaciones diplomáticas, y celebrar tratados con las potencias extranjeras, sometiéndolos á la ratificacion del congreso federal.
\item Recibir ministros y otros enviados de las potencias
extranjeras.
\item Convocar al congreso á sesiones cuando lo acuerde la diputacion permanente.
\item Facilitar al poder judicial los auxilios que necesite para el ejercicio expedito de sus funciones.
\item Habilitar toda clase de puertos, establecer aduanas marítimas y fronterizas y designar su ubicacion.
\item Conceder, conforme á las leyes, indultos á los reos sentenciados por delitos de la competencia de los tribunales federales.
\stopitemize

\subsubsection{\mbox{}} Para el despacho de los negocios del orden administrativo de la federacion, habrá el número de secretarios que establezca el congreso por una ley, la que hará la distribucion de los negocios que han de estar á cargo de cada secretaría.

\subsubsection{\mbox{}} Para ser secretario del despacho se requiere: ser ciudadano mexicano por nacimiento, estar en ejercicio de sus derechos y tener veinticinco años cumplidos.

\subsubsection{\mbox{}} Todos los reglamentos, decretos y órdenes del presidente, deberán ir firmados por el secretario del despacho encargado del ramo á que el asunto corresponde. Sin este requisito no serán obedecidos.

\subsubsection{\mbox{}} Los secretarios del despacho, luego que estén abiertas las sesiones del primer período, darán cuenta al congreso del estado de sus respectivos ramos.

\section{Del poder judicial}

\subsubsection{\mbox{}} Se deposita el ejercicio del poder judicial de la federacion en una Corte Suprema de Justicia y en los tribunales de Distrito y Circuito.

\subsubsection{\mbox{}} La Suprema Corte de Justicia se compondrá de once ministros propietarios, cuatro supernumerarios, un fiscal y un procurador general.

\subsubsection{\mbox{}} Cada uno de los individuos de la Suprema Corte de Justicia durará en su encargo seis años, y su eleccion será indirecta en primer grado, en los términos que disponga la ley electoral.

\subsubsection{\mbox{}} Para ser electo individuo de la Suprema Corte de Justicia, se necesita: estar instruido en la ciencia del derecho, á juicio de los electores, ser mayor de treinta y cinco años y ciudadano mexicano por nacimiento, en ejercicio de sus derechos.

\subsubsection{\mbox{}} Los individuos de la Suprema Corte de Justicia, al entrar á ejercer su encargo, prestarán juramento ante el congreso, y en sus recesos ante la diputacion permanente, en la forma siguiente: «¿Jurais desempeñar leal y patrióticamente el cargo de magistrado de la Suprema Corte de Justicia, que os ha conferido el pueblo, conforme á la Constitucion, y mirando en todo por el bien y prosperidad de la Union?»

\subsubsection{\mbox{}} El cargo de individuo de la Suprema Corte de Justicia solo es renunciable por causa grave, calificada por el congreso, ante quien se presentará la renuncia. En los recesos de éste la calificacion se hará por la diputacion permanente.

\subsubsection{\mbox{}} La ley establecerá y organizará los tribunales de Circuito y de Distrito.

\subsubsection{\mbox{}} Corresponde á los tribunales de la federacion, conocer:
\startitemize[R]
\item De todas las controversias que se susciten sobre el cumplimiento y aplicacion de las leyes federales.
\item De las que versen sobre derecho marítimo.
\item De aquellas en que la federacion fuere parte.
\item De las que se susciten entre dos ó más Estados.
\item De las que se susciten entre un Estado y uno ó más vecinos de otro.
\item De las del orden civil ó criminal que se susciten á consecuencia de los tratados celebrados con las potencias extranjeras.
\item De los casos concernientes á los agentes diplomáticos y cónsules.
\stopitemize

\subsubsection{\mbox{}} Corresponde á la Suprema Corte de Justicia desde la primera instancia, el conocimiento de las controversias que se susciten de un Estado con otro, y de aquellas en que la Union fuere parte.

\subsubsection{\mbox{}} Corresponde tambien á la Suprema Corte de Justicia dirimir las competencias que se susciten entre los tribunales de la federación; entre éstos y los de los Estados, ó entre los de un Estado y los de otro.

\subsubsection{\mbox{}} En los demás casos comprendidos en el artículo 97, la Suprema Corte de Justicia será tribunal de apelacion, ó bien de última instancia, conforme á la graduacion que haga la ley de las atribuciones de los tribunales de Circuito y de Distrito.

\subsubsection{\mbox{}} Los tribunales de la federacion resolverán toda controversia que se suscite:
\startitemize[R]
\item Por leyes ó actos de cualquiera autoridad que violen las garantías individuales.
\item Por leyes ó actos de la autoridad federal que vulneren ó restrinjan la soberanía de los Estados.
\item Por las leyes ó actos de las autoridades de éstos, que invadan la esfera de la autoridad federal.
\stopitemize

\subsubsection{\mbox{}} Todos los juicios de que habla el artículo anterior se seguirán, á peticion de la parte agraviada, por medio de procedimientos y formas del órden jurídico, que determinará una ley. La sentencia será siempre tal, que solo se ocupe de individuos particulares, limitándose á protegerlos y ampararlos en el caso especial sobre que verse el proceso, sin hacer ninguna declaracion general respecto de la ley ó acto que la motivare.

\chapter{De la responsabilidad de los funcionarios públicos}

\subsubsection{\mbox{}} Los diputados al congreso de la Union, los individuos de la Suprema Corte de Justicia y los Secretarios del Despacho, son responsables por los delitos comunes que cometan durante el tiempo de su encargo, y por los delitos, faltas ú omisiones en que incurran en el ejercicio de ese mismo encargo. Los gobernadores de los Estados lo son igualmente por la infraccion de la Constitucion y leyes federales. Lo es tambien el presidente de la República; pero durante el tiempo de su encargo solo podrá ser acusado por los delitos de traicion á la patria, violacion expresa de la Constitucion, ataque á la libertad electoral y delitos graves del órden comun.

\subsubsection{\mbox{}} Si el delito fuere comun, el congreso erigido en gran jurado declarará, á mayoría absoluta de votos, si ha ó no lugar á proceder contra el acusado. En caso negativo no habrá lugar á ningun procedimiento ulterior. En el afirmativo, el acusado queda por el mismo hecho, separado de su encargo y sujeto á la accion de los tribunales comunes.

\subsubsection{\mbox{}} De los delitos oficiales conocerán el congreso como jurado de acusación, y la Suprema Corte de Justicia como jurado de sentencia.

El jurado de acusacion tendrá por objeto declarar á mayoría absoluta de votos, si el acusado es ó no culpable. Si la declaracion fuere absolutoria, el funcionario continuará en el ejercicio de su encargo. Si fuere condenatoria, quedará inmediatamente separado de dicho encargo, y será puesto á disposicion de la Suprema Corte de Justicia. Esta, en tribunal pleno, y erigida en jurado de sentencia, con audiencia del reo, del fiscal y del acusador, si lo hubiere, procederá á aplicar, á mayoría absoluta de votos, la pena que la ley designe.

\subsubsection{\mbox{}} Pronunciada una sentencia de responsabilidad por delitos oficiales, no puede concederse al reo la gracia de indulto.

\subsubsection{\mbox{}} La responsabilidad por delitos y faltas oficiales solo podrá exigirse durante el período en que el funcionario ejerza su encargo y un año despues.

\subsubsection{\mbox{}} En demandas del órden civil no hay fuero, ni inmunidad para ningun funcionario público.

\chapter{De los Estados de la federacion}

\subsubsection{\mbox{}} Los Estados adoptarán para su régimen interior forma de gobierno republicano representativo popular.

\subsubsection{\mbox{}} Los Estados pueden arreglar entre sí, por convenios amistosos, sus respectivos límites; pero no se llevarán á efecto esos arreglos sin la aprobacion del congreso de la Unión.

\subsubsection{\mbox{}} Los Estados no pueden en ningun caso:
\startitemize[R]
\item Celebrar alianza, tratado ó coalicion con otro Estado, ni con potencias extranjeras. Exceptúase la coalición, que pueden celebrar los Estados fronterizos, para la guerra ofensiva ó defensiva contra los bárbaros.
\item Expedir patentes de corso ni de represalias.
\item Acuñar moneda, emitir papel moneda, ni papel sellado.
\stopitemize

\subsubsection{\mbox{}} Tampoco pueden, sin consentimiento del congreso de la Union:
\startitemize[R]
\item Establecer derechos de tonelaje ni otro alguno de puerto; ni imponer contribuciones ó derechos sobre importaciones ó exportaciones.
\item Tener en ningun tiempo tropa permanente, ni buques de guerra.
\item Hacer la guerra por sí á alguna potencia extranjera. Exceptúanse los casos de invasion ó de peligro tan inminente que no admita demora. En estos casos darán cuenta inmediatamente al presidente de la República.
\stopitemize

\subsubsection{\mbox{}} Cada Estado tiene obligacion de entregar sin demora los criminales de otros Estados á la autoridad que los reclame.

\subsubsection{\mbox{}} Los gobernadores de los Estados están obligados á publicar y hacer cumplir las leyes federales.

\subsubsection{\mbox{}} En cada Estado de la federacion se dará entera fé y crédito á los actos públicos, registros y procedimientos judiciales de todos los otros. El congreso puede, por medio de leyes generales, prescribir la manera de probar dichos actos, registros y procedimientos y el efecto de ellos.

\subsubsection{\mbox{}} Los poderes de la Union tienen el deber de proteger á los Estados contra toda invasion ó violencia exterior. En caso de sublevacion ó trastorno interior, les prestarán igual proteccion, siempre que sean excitados por la legislatura del Estado ó por su ejecutivo, si aquella no estuviere reunida.

\chapter{Prevenciones generales}

\subsubsection{\mbox{}} Las facultades que no están expresamente concedidas por esta Constitucion á los funcionarios federales, se entienden reservadas á los Estados.

\subsubsection{\mbox{}} Ningun individuo puede desempeñar á la vez dos cargos de la Union de eleccion popular; pero el nombrado puede elegir entre ambos el que quiera desempeñar.

\subsubsection{\mbox{}} Ningun pago podrá hacerse que no esté comprendido en el presupuesto ó determinado por ley posterior.

\subsubsection{\mbox{}} El presidente de la República, los individuos de la Suprema Corte de Justicia, los diputados y demás funcionarios públicos de la federacion, de nombramiento popular, recibirán una compensacion por sus servicios, que será determinada por la ley y pagada por el tesoro federal. Esta compensacion no es renunciable, y la ley que la aumente ó la disminuya, no podrá tener efecto durante el período en que un funcionario ejerce el cargo.

\subsubsection{\mbox{}} Todo funcionario público, sin excepcion alguna, ántes de tomar posesion de su encargo, prestará juramento de guardar esta Constitucion y las leyes que de ella emanen.

\subsubsection{\mbox{}} En tiempo de paz ninguna autoridad militar puede ejercer más funciones, que las que tengan exacta conexion con la disciplina militar. Solamente habrá comandancias militares fijas y permanentes en los castillos, fortalezas y almacenes que dependan inmediatamente del gobierno de la Union; ó en los campamentos, cuarteles ó depósitos que, fuera de las poblaciones, estableciere para la estacion de las tropas.

\subsubsection{\mbox{}} Corresponde exclusivamente á los poderes federales ejercer, en materias de culto religioso y disciplina externa, la intervencion que designen las leyes.

\subsubsection{\mbox{}} Para el dia 1o de Junio de 1858 quedarán abolidas las alcabalas y aduanas interiores en toda la República.

\subsubsection{\mbox{}} Estarán bajo la inmediata inspeccion de los poderes federales los fuertes, cuarteles, almacenes de depósitos y demás edificios necesarios al gobierno de la Union.

\subsubsection{\mbox{}} Esta Constitución, las leyes del congreso de la Union que emanen de ella y todos los tratados hechos ó que se hicieren por el presidente de la República, con aprobacion del congreso; serán la ley suprema de toda la Union. Los jueces de cada Estado se arreglarán á dicha Constitucion, leyes y tratados, á pesar de las disposiciones en contrario que pueda haber en las constituciones ó leyes de los Estados.

\chapter{De la reforma de la Constitucion}

\subsubsection{\mbox{}} La presente Constitucion puede ser adicionada ó reformada. Para que las adiciones ó reformas lleguen á ser parte de la Constitucion, se requiere que el congreso de la Union, por el voto de las dos terceras partes de sus individuos presentes, acuerde las reformas ó adiciones, y que éstas sean aprobadas por la mayoría de las legislaturas de los Estados. El congreso de la Union hará el cómputo de los votos de las legislaturas y la declaracion de haber sido aprobadas las adiciones ó reformas.

\chapter{De la inviolabilidad de la Constitucion}

\subsubsection{\mbox{}} Esta Constitucion no perderá su fuerza y vigor, aun cuando por alguna rebelion se interrumpa su observancia. En caso de que por un trastorno público se establezca un gobierno contrario á los principios que ella sanciona, tan luego como el pueblo recobre su libertad, se restablecerá su observancia, y con arreglo á ella y á las leyes que en su virtud se hubieren expedido, serán juzgados, así los que hubieren figurado en el gobierno emanado de la rebelion, como los hubieren cooperado á ésta.

\subsubsubject{Artículo transitorio} Esta Constitucion se publicará desde luego y será jurada con la mayor solemnidad en toda la República; pero con excepcion de las disposiciones relativas á las elecciones de los supremos poderes federales y de los Estados, no comenzará á regir hasta el dia 16 de Setiembre próximo venidero, en que debe instalarse el primer congreso constitucional. Desde entónces el presidente de la República y la Suprema Corte de Justicia, que deben continuar en ejercicio hasta que tomen posesion los individuos electos constitucionalmente, se arreglarán en el desempeño de sus obligaciones y facultades á los preceptos de la Constitucion.

%\page[right,empty,right]
\page[yes]

Dada en el salon de sesiones del congreso, en México, á cinco de Febrero de mil ochocientos cincuenta y siete, trigésimo sétimo de la Independencia.-

\blank\startxtable
\startxrow
    \startxcell Valentin Gomez Farías, \stopxcell
    \startxcell diputado por el Estado de Jalisco, Presidente.- \stopxcell
\stopxrow
\startxrow[bottomframe=on]
    \startxcell Leon Guzman, \stopxcell
    \startxcell diputado por el Estado de México, Vicepresidente.- \stopxcell
\stopxrow
\startxrow
    \startxcell Por el Estado de Aguascalientes, \stopxcell
    \startxcell Manuel Buenrostro.- \stopxcell
\stopxrow
\startxrow
    \startxcell Por el Estado de Chiapas, \stopxcell
    \startxcell Francisco Robles, Matías Castellanos.- \stopxcell
\stopxrow
\startxrow
    \startxcell Por el Estado de Chihuahua, \stopxcell
    \startxcell José Eligio Muñoz, Pedro Ignacio Irigoyen.- \stopxcell
\stopxrow
\startxrow
    \startxcell Por el Estado de Coahuila, \stopxcell
    \startxcell Simon de la Garza y Melo.- \stopxcell
\stopxrow
\startxrow
    \startxcell Por el Estado de Durango, \stopxcell
    \startxcell Marcelino Castañeda, Francisco Zarco.- \stopxcell
\stopxrow
\startxrow
    \startxcell Por el Distrito federal, \stopxcell
    \startxcell Francisco de P. Cendejas, José María del Rio,
    Ponciano Arriaga, J. M. del Castillo Velasco,
    Manuel Morales Puente.- \stopxcell
\stopxrow
\startxrow
    \startxcell Por el Estado de Guanajuato, \stopxcell
    \startxcell Ignacio Sierra, Antonio Lémus, José de la Luz Rosas, \\
    Juan Morales, Antonio Aguado, Francisco P. Montañez, \\
    Francisco Guerrero, Blas Balcárcel.- \stopxcell
\stopxrow
\startxrow
    \startxcell Por el Estado de Guerrero, \stopxcell
    \startxcell Francisco Ibarra.- \stopxcell
\stopxrow
\startxrow
    \startxcell Por el Estado de Jalisco, \stopxcell
    \startxcell Espiridion Moreno, Mariano Torres Aranda, \\
    Jesus Anaya y Hermosillo, Albino Aranda, \\
    Ignacio Luis Vallarta, Benito Gomez Farías, \\
    Jesus D. Rojas, Ignacio Ochoa Sanchez, \\
    Guillermo Langlois, Joaquin M. Degollado.- \stopxcell
\stopxrow
\startxrow
    \startxcell Por el Estado de México, \stopxcell
    \startxcell Antonio Escudero, José L. Revilla, Julian Estrada, \\
    I. de la Peña y Barragan, Estéban Paez, \\
    Rafael María Villagran, Francisco Fernandez de Alfaro, \\
    Justino Fernandez, Eulogio Barrera, \\
    Manuel Romero Rubio, Manuel de la Peña y Ramirez, \\
    Manuel Fernandez Soto.- \stopxcell
\stopxrow
\startxrow
    \startxcell Por el Estado de Michoacán, \stopxcell
    \startxcell Santos Degollado, Sabás Iturbide, Francisco G. Anaya, \\
    Ramon I. Alcaraz, Francisco Diaz Barriga,\\
    Luis Gutierrez Correa, Mariano Ramirez, \\
    Mateo Echaiz.- \stopxcell
\stopxrow
\startxrow
    \startxcell Por el Estado de Nuevo Leon, \stopxcell
    \startxcell Manuel P. de Llano.- \stopxcell
\stopxrow
\startxrow
    \startxcell Por el Estado de Oaxaca, \stopxcell
    \startxcell Mariano Zavala, G. Larrazabal, Ignacio Mariscal, \\
    Juan N. Cerqueda, Félix Romero, Manuel E. Goytia.- \stopxcell
\stopxrow
\startxrow
    \startxcell Por el Estado de Puebla, \stopxcell
    \startxcell Miguel María Arrioja, Fernando María Ortega, \\
    Guillermo Prieto, J. Mariano Viadas, Francisco Banuet, \\
    Manuel M. Bargas, Francisco Lazo Estrada, \\
    Juan N. Ibarra, Juan N. de la Parra.- \stopxcell
\stopxrow
\startxrow
    \startxcell Por el Estado de Querétaro, \stopxcell
    \startxcell Ignacio Reyes.- \stopxcell
\stopxrow
\startxrow
    \startxcell Por el Estado de San Luis Potosí, \stopxcell
    \startxcell Francisco J. Villalobos, Pablo Tellez.- \stopxcell
\stopxrow
\startxrow
    \startxcell Por el Estado de Sinaloa, \stopxcell
    \startxcell Ignacio Ramirez.- \stopxcell
\stopxrow
\startxrow
    \startxcell Por el Estado de Sonora, \stopxcell
    \startxcell Benito Quintana.- \stopxcell
\stopxrow
\startxrow
    \startxcell Por el Estado de Tabasco, \stopxcell
    \startxcell Gregorio Payró.- \stopxcell
\stopxrow
\startxrow
    \startxcell Por el Estado de Tamaulipas, \stopxcell
    \startxcell Luis García de Arellano.- \stopxcell
\stopxrow
\startxrow
    \startxcell Por el Estado de Tlaxcala, \stopxcell
    \startxcell José Mariano Sanchez.- \stopxcell
\stopxrow
\startxrow
    \startxcell Por el Estado de Veracruz, \stopxcell
    \startxcell José de Empáran, José María Mata,\\
    Rafael Gonzalez Paez, Mariano Vega.- \stopxcell
\stopxrow
\startxrow
    \startxcell Por el Estado de Yucatan, \stopxcell
    \startxcell Benito Quijano, Francisco Iniestra, Pedro de Baranda,\\
    Pedro Contreras Elizalde.- \stopxcell
\stopxrow
\startxrow
    \startxcell Por el Territorio de Tehuantepec, \stopxcell
    \startxcell Joaquin García Granados.- \stopxcell
\stopxrow
\startxrow
    \startxcell Por el Estado de Zacatecas, \stopxcell
    \startxcell Miguel Auza, Agustin López de Nava,\\
    Basilio Perez Gallardo.- \stopxcell
\stopxrow
\startxrow
    \startxcell Por el Territorio de la Baja California, \stopxcell
    \startxcell Mateo Ramirez.-\stopxcell
\stopxrow
\stopxtable\blank

\blank\startxtable
\startxrow
    \startxcell José María Cortés y Esparza, \stopxcell
    \startxcell por el Estado de Guanajuato, \stopxcell
    \startxcell diputado secretario.- \stopxcell
\stopxrow
\startxrow
    \startxcell Isidoro Olvera, \stopxcell
    \startxcell por el Estado de Mexico, \stopxcell
    \startxcell diputado secretario.- \stopxcell
\stopxrow
\startxrow
    \startxcell Juan de Dios Arias, \stopxcell
    \startxcell por el Estado de Puebla, \stopxcell
    \startxcell diputado secretario.- \stopxcell
\stopxrow
\startxrow
    \startxcell J. A. Gamboa, \stopxcell
    \startxcell por el Estado de Oaxaca, \stopxcell
    \startxcell diputado secretario.\stopxcell
\stopxrow
\stopxtable\blank

Por tanto, mando se imprima, publique, circule y se le dé el debido cumplimiento, en los términos que ella prescribe. Palacio del gobierno nacional en México, Febrero doce de mil ochocientos cincuenta y siete.-
Ignacio Comonfort.-
Al ciudadano Ignacio de la Llave, secretario de Estado y del despacho de Gobernacion.

Y lo comunico á Vd. para su publicacion y cumplimiento.

Dios y libertad. México, 12 de Febrero de 1857.- Llave.

\completecontent[criterium=previous]

\page[last,quadruple]

\stoptext
